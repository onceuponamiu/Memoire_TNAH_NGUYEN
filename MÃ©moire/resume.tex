\chapter*{Résumé}
\medskip
Ce mémoire explore l'automatisation de la création des tables des matières pour les débats parlementaires aux archives du Sénat, en se basant sur les Journaux Officiels de 2022. Le projet, réalisé dans le cadre d'un stage de six mois, vise à simplifier la gestion des métadonnées grâce à des outils d'intelligence artificielle et de traitement du langage naturel. Ce mémoire souligne l'importance de rendre les données accessibles tout en réduisant le temps de traitement manuel.
Grâce à des bibliothèques Python comme pdfplumber et des outils de traitement de texte comme Regex, il a été possible d'extraire automatiquement des informations pertinentes de documents PDF, comme le nom des intervenants et les textes de lois. Bien que certaines limites techniques, comme la reconnaissance d'entités spécifiques et la qualité des données sources, aient persistées, les résultats démontrent une nette amélioration en termes de précision et de rapidité par rapport à une saisie manuelle.
Le mémoire conclut que cette méthode d’automatisation ouvre des perspectives intéressantes pour le futur, notamment pour d'autres corpus de données parlementaires. Cependant, des avancées techniques supplémentaires, comme l'intégration de modèles d'apprentissage plus sophistiqués, et l'utilisation de matériel plus optimisés pour le traitement de l'IA, seraient nécessaires pour surmonter les défis actuels. .\\
	
\textbf{Mots-clés:} automatisation; intelligence artificielle; traitement du langage naturel; Sénat; Journaux Officiels; tables des matières; Python; extraction de texte, pdfplumber; regex.
	
\textbf{Informations bibliographiques:} Le Thuy Tien NGUYEN, \textit{Donner accès aux débats parlementaires au Sénat. Les enjeux de l’automatisation de la création des tables des matières pour les sessions parlementaires. L’exemple des Journaux Officiels 2022.}, mémoire de master \enquote{Technologies numériques appliquées à l'histoire}, directrice Marie PUREN, École nationale des chartes, 2024.