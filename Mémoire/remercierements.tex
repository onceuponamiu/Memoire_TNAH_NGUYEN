\chapter*{Remerciements}

Un mémoire est un parcours intellectuel exigeant et parfois solitaire, mais heureusement pour moi, j'ai eu la chance d'avoir de nombreux professeurs, directeurs, proches et amis qui ont tous contribué à enrichir ce projet et à stimuler ma curiosité. 
Je souhaite tout d'abord remercier mes professeurs à l'ENC-PSL, qui m'ont nourri avec des théories et des méthodes pratiques me permettant d'accéder aux moyens techniques dont j'avais besoin, tout en éclairant ma pensée par des connaissances élargies. J'ai acquis des outils pour m'entraîner et pratiquer au cours de ce stage, outils qui constituent maintenant les clés de ma future carrière.

Je remercie également Jean-Marc TICHI et Laurent BAOUR, qui m'ont accueilli en stage aux Archives du Sénat, notamment pour ce projet d'automatisation des Tables du Sénat. Je tenais à communiquer une mention toute spéciale à Laurent BAOUR, mon maître de stage, qui a toujours été présent pour m'aider à développer mes compétences durant ce stage. Il m'a prodigué de nombreux conseils et explications sur les bases du Sénat, mais également sur la science politique et a été mon lien pour échanger avec toute l'équipe des archivistes. Je remercie aussi les équipes opérationnelles, qui réalisent l'analyse et le bornage des Journaux Officiels, de m'avoir ouvert leurs portes et généreusement partagé leur vision de ce travail et leurs besoins professionnels. Échanger avec eux m'a ainsi permis de mieux comprendre leurs attentes et donner un sens plus humain à ma mission d'automatisation.

Bien entendu, je souhaite remercier ma directrice de mémoire, Marie PUREN, qui m'a guidé dans le développement de mon corpus ainsi que de mon livrable technique. J'ai apprécié tout particulièrement l'équilibre qu'elle a su établir entre la résolutions des enjeux du stage et ma liberté d'explorer les méthodes techniques qui me semblaient les plus appropriées. Sans ses commentaires avisés et ses références de qualité, mon mémoire aurait sans doute pris beaucoup plus de temps à atteindre un tel niveau de rigueur scientifique. Au-delà de son accompagnement, je lui suis très reconnaissante pour sa patience et sa tolérance.

Je tiens également à témoigner ma gratitude à Emmanuelle BERMES, responsable de master, qui m'a fait confiance et sans qui je n'aurais pas pu rejoindre ce parcours académique. Elle m'a soutenue et conseillée tout au long de cette année, toujours à l'écoute et prête à me rassurer, elle a su, à travers nos échanges, raviver ma motivation, m'aidant à entrevoir de nouvelles perspectives à mes études ainsi que sur mes choix. Ce travail n'aurait pas pu être réalisé sans son soutien moral et sa confiance. Je lui suis également très reconnaissante d'avoir tenu compte de mes besoins d'apprentissage aussi bien académiques que émotionnels. Sa capacité à comprendre ma situation, en tant qu'expatriée, m'a permis de garder ma motivation dans les moments où j'en avais le plus besoin.

Je suis également très reconnaissante envers ma famille, qui m'a soutenu tant sur le plan matériel que moral, malgré la distance.
Je remercie également mes amis, Giang et Vy, qui m'ont soutenu et encouragé avec intérêt. Je remercie aussi Hưng et Guillaume, qui ont gentiment accepté de m'aider à débugguer mon code à plusieurs reprises. Je suis reconnaissante envers Thomas, qui m'a écoutée avec patience, qui a toléré mes moments de morosité, sans jamais cesser de m'encourager à communiquer ce que je ressentais, merci de toujours être là.